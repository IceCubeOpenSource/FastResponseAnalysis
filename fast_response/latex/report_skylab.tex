\documentclass[titlepage]{article}
%\usepackage[printwatermark]{xwatermark}
\usepackage{tikz}
\usepackage{xcolor}
\usepackage{graphicx}
\usepackage{grffile}
\usepackage{longtable}
\usepackage[margin=1.0in]{geometry}
%\usepackage{lipsum}
%\usepackage{gensymb}

%\newwatermark*[allpages,color=red!25,angle=45,scale=3,xpos=0,ypos=0]{TEST}

%\savebox\mybox{\tikz[color=red,opacity=0.5]\node{TEST};}
%\newwatermark*[
%  allpages,
%  angle=45,
%  scale=12,
%  xpos=-40,
%  ypos=50
%]{\usebox\mybox}

\setlength{\parindent}{0cm} 
\include{r}

\newcommand{\degree}{$^{\circ}$}
\begin{document}

\begin{titlepage}
  \centering
  \vspace{4cm}
  {\huge\bfseries IceCube Fast-Response \\ Analysis Report\par}
  \vspace{1cm}
  {\LARGE For Internal Use Only\par}
  \vfill
  {\Large Source Name: \\ \itshape\sourcename\par}
  \vspace{0.5cm}
  {\Large Observation Date(s):\\ \obsdate \par}
  \vfill
  \vspace{1cm} 
  {\Large Report Generated On:\\ \reportdate \par}
% {\color{red} This is a test. \\
%   This station is conducting a test of the IceCube Alert System.\\
%   This is only a test. }
\end{titlepage}

\section{Inputs}
\subsection{Source Information}
\sourcetable
\subsection{On-Time Data}
\ontimetable
\subsection{Skylab Analysis Information}
\skylabtable 
% Information about skylab stable release, path to skylab, etc.

\pagebreak
\section{Detector Operations}

\subsection{Run Times}
\runtimetable

Total Livetime = \livetime\,s

\subsection{Run Status}
\runstatustable

\pagebreak
\subsection{Event Rates}
Plots for key trigger and filter rates for the data period
and the neighboring runs.  Blue indicates selected time window.
Badness $>10$s may indicate a problem.

\vspace{1em}
{
 \centering        
 \includegraphics[width=1.0\textwidth]{\multiplicity}
 \includegraphics[width=1.0\textwidth]{\badnessplot}
 \includegraphics[width=1.0\textwidth]{\muonfilter}
 \includegraphics[width=1.0\textwidth]{\Lfilter}
 \includegraphics[width=1.0\textwidth]{\gfurate}
 % \includegraphics[width=1.0\textwidth]{"/home/rhussain/icecube/dump/badness_plot.png"}
}


%\pagebreak
%\section{Energy PDF}
%Ratio of Energy PDFs from the selected Zenith band.
%
%\energytable
%
%{
%  \centering
%
%    {\Large Energy PDF Fit}
%
%    \includegraphics[width=0.8\textwidth]{\analysisid_EnergyPDFRatio}
%
%    {\Large Energy of Observed Events}
%
%    \includegraphics[width=0.8\textwidth]{\analysisid_energy_events}
%
%    }

\pagebreak
\section{Results}

{
  \centering
  {\Large All Sky On-time Events}

  \includegraphics[width=0.9\textwidth]{\skymap}

  \includegraphics[width=0.9\textwidth]{\skymapzoom}
  
}
\pagebreak


\subsection{Coincident Events}
For point sources, any events from the relevant time window with a spatial times energy weight greater than 10 is included in this table. For skymaps, any event reconstructed within the 90\% contour is included (for longer timescale cascades we skip this because of the large localization). 
\event

\subsection{Likelihood Analysis}
\results
\pagebreak

\backgroundpdfplot

\survivialfunctionplot

\pagebreak
{
  \centering

  {\Large Limit $dN/dE$}

  \includegraphics[width=0.9\textwidth]{\limitdNdE}
  \\
  \\

  \tsd

  \\
  \\
  \upperlim

  \\
  \\
  \nsscan

  These plots are meant as validations of the minimizer, and cannot be interpreted directly using Wilk's assuming one degree of freedom because the low statistics invalidates the asymptotic condition for Wilk's theorem.  
}

\vfill        
%{\color{red}If this had been an actual analysis,
%  the attention signal you just heard would have been
%  followed by official information, news, or instructions.}

\end{document}
